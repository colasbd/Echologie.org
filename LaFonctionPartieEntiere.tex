%% LyX 2.2.1 created this file.  For more info, see http://www.lyx.org/.
%% Do not edit unless you really know what you are doing.
\documentclass[french,12pt]{paper}
\usepackage[T1]{fontenc}
\usepackage[utf8]{inputenc}
\usepackage[landscape,a4paper]{geometry}
\geometry{verbose,tmargin=31mm,bmargin=28mm,lmargin=27mm,rmargin=27mm}
\pagestyle{empty}
\setlength{\parskip}{0bp}
\setlength{\parindent}{0pt}
\usepackage{fancybox}
\usepackage{calc}
\usepackage{mathtools}
\usepackage{amsmath}
\usepackage{amssymb}
\usepackage{setspace}
\onehalfspacing

\makeatletter
%%%%%%%%%%%%%%%%%%%%%%%%%%%%%% User specified LaTeX commands.
\usepackage{tikz}
\usetikzlibrary{calc}
\usetikzlibrary{shapes}
\usepackage{mathrsfs}
\usepackage{mathabx}
\usepackage{txfonts}
\usepackage{pxfonts}
\usepackage{titling}
\usepackage{array}
%\usepackage{yhmath}

\newdimen\un \un=1mm
\def\bordure{
\begin{tikzpicture}[overlay,remember picture]
\node(Triskell) at ($(current page.north west)+(21*\un,-20*\un)$){};
\def\p{.75}
\def\ang{45}
\def\alp{160.2}
\def\bet{72.42}
\def\gam{-13.2}
\draw [very thick, line width = 8pt, color = red!25!blue!33.333!green!50]
($(current page.north west)+(20*\un,-20*\un)$)
-- ($(current page.north east)+(-24*\un,-20*\un)$)
node [draw, ellipse, fill, text=white, pos=.5] {\thetitle}
arc (90 : 0 : 4*\un)
-- ($(current page.south east)+(-20*\un,24*\un)$)
node[text=white,pos=.5 , rotate=90]
{\tiny Ce document est sous licence GNU FDL,
 il est librement modifiable et distribuable.
 Sources et licence complètes disponible sur le site.
 Copyright 2015, Jean-Christophe Jameux}
arc (0 : -90 : 4*\un)
-- ($(current page.south west)+(24*\un,20*\un)$)
node[draw, ellipse, fill, text=white, pos=.15] {\bf Echologie.org}
arc (-90 : -180 : 4*\un)
-- ($(current page.north west)+(20*\un,-20*\un)$);
\draw [fill = white, color = red!25!blue!33.333!green!50]
(Triskell) + (1.2*\un,-6*\un) circle (15*\un);
\draw [fill = white, color = white] (Triskell) circle (2*\un);
\draw [fill = white, color = white]
(Triskell) + ({120*(1+\p)} : 3*\un)
arc ({120*(1+\p)} : 120 : 3*\un)
arc (180+\ang : 180-\ang :3*\un)
arc (\alp-\ang : \alp+\ang+24.8 : 5*\un);
\draw [fill = white, color = white]
(Triskell) + (120*\p : 3*\un)
arc (120*\p : 0 : 3*\un)
arc (90+\ang : 90-\ang : 6*\un)
arc (\bet-\ang : \bet+\ang+5.9 : 8*\un);
\draw [fill = white, color = white]
(Triskell) + ({120*(2+\p)} : 3*\un)
arc ({120*(2+\p)} : 240 : 3*\un)
arc (\ang : -\ang : 12*\un)
arc (\gam-\ang : \gam+\ang+.85 : 13*\un);
\end{tikzpicture}}

\AtBeginDocument{
  \def\labelitemi{\Large\(\bullet\)}
}

\makeatother

\usepackage{babel}
\makeatletter
\addto\extrasfrench{%
   \providecommand{\og}{\leavevmode\flqq~}%
   \providecommand{\fg}{\ifdim\lastskip>\z@\unskip\fi~\frqq}%
}

\makeatother
\begin{document}
\title{\large\bf La fonction Partie Entière}

\textbf{\large{}\hphantom{\textbf{\large{}Deca.}}On se propose de
démontrer la proposition $\forall x\in\mathbb{R},\exists!n\in\mathbb{Z},n\leq x<n+1$}\\
\textbf{\large{}\hphantom{\textbf{\large{}De..}}L'unique $n$ dont
on affirme l'existence pour tout réel $x$ sera appelé la partie entière
de $x$ et sera noté $E\left(x\right)$}\\
\textbf{\large{}\hphantom{\textbf{\large{}D}}Comme $\mathbb{R}=\mathbb{R}_{+}\cup\mathbb{R}_{-}$,
cela revient a montré les trois propositions suivantes :\vspace{-2mm}
}{\large \par}
\begin{itemize}
\item \textbf{\large{}$\forall x\in\mathbb{R}_{+},\exists n\in\mathbb{Z},n\leq x<n+1$}{\large \par}
\item \textbf{\large{}$\forall x\in\mathbb{R}_{-},\exists n\in\mathbb{Z},n\leq x<n+1$}{\large \par}
\item \textbf{\large{}$\forall x\in\mathbb{R},\forall n_{1},n_{2}\in\mathbb{Z},(n_{1}\leq x<n_{1}+1\text{ et }n_{2}\leq x<n_{2}+1)\Longrightarrow n_{1}=n_{2}$
\quad{}(c'est la condition d'unicité)}{\large \par}
\end{itemize}
\textbf{\large{}}%
{\fboxsep 3mm\Ovalbox{\begin{minipage}[t]{1\columnwidth - 2\fboxsep - 1.6pt}%
\textbf{\large{}Soit $x\in\mathbb{R}_{+}$, on sait par la propriété
d'Archimède que l'ensemble $A=\{r\in\mathbb{N}|r>x\}$ est non vide.
Il admet donc un plus petit élément $r$ vérifiant $r>x$ et tel que
$r-1\leq x$. En posant $n:=r-1$, on a montré l'existence d'un $n\in\mathbb{Z}$
vérifiant $n\leq x<n+1$}%
\end{minipage}}}{\large \par}

\textbf{\large{}\vfill{}
}%
{\fboxsep 3mm\Ovalbox{\begin{minipage}[t]{1\columnwidth - 2\fboxsep - 1.6pt}%
\textbf{\large{}Soit maintenant $x\in\mathbb{R_{-}}$, on a de même
par la propriété d'Archimède que l'ensemble $A=\{r\in\mathbb{N}|r\geq-x\}$
est non vide. Il admet donc aussi un plus petit élément $r$ vérifiant
$r\geq-x$ et tel que $r-1<-x$. En posant $n\coloneqq-r$, on a montré
l'existence d'un $n\in\mathbb{Z}$ vérifiant $n\leq x<n+1$}%
\end{minipage}}}{\large \par}

\textbf{\large{}\vfill{}
}%
{\fboxsep 3mm\Ovalbox{\begin{minipage}[t]{1\columnwidth - 2\fboxsep - 1.6pt}%
\textbf{\large{}Enfin, soient $x\in\mathbb{R}$ et $n_{1},n_{2}\in\mathbb{Z}$
tels que $n_{1}\leq x<n_{1}+1$ et $n_{2}\leq x<n_{2}+1$. En réécrivant
la deuxième inégalité sous la forme $-n_{2}-1<x\leq-n_{2}$ et en
l'additionnant à la première, on obtient l'inégalité $(n_{1}-n_{2})-1<0<(n_{1}-n_{2})+1$
qui affirme que l'entier $(n_{1}-n_{2})$ est à une distance strictement
inférieure à $1$ de $0$, il est donc nul. Ainsi $n_{1}=n_{2}$}%
\end{minipage}}}{\large \par}

\bordure
\end{document}
