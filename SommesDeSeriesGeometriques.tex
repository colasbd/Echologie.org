%% LyX 2.2.1 created this file.  For more info, see http://www.lyx.org/.
%% Do not edit unless you really know what you are doing.
\documentclass[french,english,12pt]{paper}
\usepackage[T1]{fontenc}
\usepackage[latin9]{inputenc}
\usepackage[a4paper]{geometry}
\geometry{verbose,tmargin=37mm,bmargin=0mm,lmargin=27mm,rmargin=27mm}
\pagestyle{empty}
\setlength{\parskip}{\medskipamount}
\setlength{\parindent}{0pt}
\usepackage{amsmath}
\usepackage{amssymb}
\usepackage{setspace}

\makeatletter
%%%%%%%%%%%%%%%%%%%%%%%%%%%%%% User specified LaTeX commands.
\usepackage{tikz}
\usetikzlibrary{calc}
\usetikzlibrary{shapes}
\usepackage{mathrsfs}
\usepackage{mathabx}
\usepackage{txfonts}
\usepackage{pxfonts}
\usepackage{titling}
\usepackage{array}
%\usepackage{yhmath}

\newdimen\un \un=.94mm
\def\bordure{
\begin{tikzpicture}[overlay,remember picture]
\def\p{.75}
\def\ang{45}
\def\alp{160.2}
\def\bet{72.42}
\def\gam{-13.2}
\draw [very thick, line width = 8pt, color = red!25!blue!33.333!green!50]
($(current page.north west)+(20*\un,-20*\un)$)
-- ($(current page.north east)+(-24*\un,-20*\un)$)
node [draw, ellipse, fill, text=white, pos=.53] {\thetitle}
arc (90 : 0 : 4*\un)
-- ($(current page.south east)+(-20*\un,24*\un)$)
node[text=white,pos=.65 , rotate=90]
{\tiny Ce document est sous licence GNU FDL,
 il est librement modifiable et distribuable.
 Licence et sources compl�tes disponibles sur le site.
 Copyright 2015, Jean-Christophe Jameux}
arc (0 : -90 : 4*\un)
-- ($(current page.south west)+(24*\un,20*\un)$)
node[draw, ellipse, fill, text=white, pos=.23] {\bf\large Echologie.org}
arc (-90 : -180 : 4*\un)
-- ($(current page.north west)+(20*\un,-20*\un)$);

\un=1mm
\node(Triskell) at ($(current page.north west)+(19*\un,-18*\un)$){};
\draw [fill = white, color = red!25!blue!33.333!green!50]
(Triskell) + (1.2*\un,-6*\un) circle (15*\un);
\draw [fill = white, color = white] (Triskell) circle (2*\un);
\draw [fill = white, color = white]
(Triskell) + ({120*(1+\p)} : 3*\un)
arc ({120*(1+\p)} : 120 : 3*\un)
arc (180+\ang : 180-\ang :3*\un)
arc (\alp-\ang : \alp+\ang+24.8 : 5*\un);
\draw [fill = white, color = white]
(Triskell) + (120*\p : 3*\un)
arc (120*\p : 0 : 3*\un)
arc (90+\ang : 90-\ang : 6*\un)
arc (\bet-\ang : \bet+\ang+5.9 : 8*\un);
\draw [fill = white, color = white]
(Triskell) + ({120*(2+\p)} : 3*\un)
arc ({120*(2+\p)} : 240 : 3*\un)
arc (\ang : -\ang : 12*\un)
arc (\gam-\ang : \gam+\ang+.85 : 13*\un);
\end{tikzpicture}}

\makeatother

\usepackage{babel}
\makeatletter
\addto\extrasfrench{%
   \providecommand{\og}{\leavevmode\flqq~}%
   \providecommand{\fg}{\ifdim\lastskip>\z@\unskip\fi~\frqq}%
}

\makeatother
\begin{document}
\selectlanguage{french}%
\begin{onehalfspace}
\title{\Large\bf Sommes de s�ries g�om�triques}
\end{onehalfspace}

\selectlanguage{english}%
\textbf{\Large{}$\bullet$ Soient $q\in\mathbb{R}$ et $v$ une suite
g�om�trique de raison $q$ d�finie sur $\mathbb{N}$, i.e. v�rifiant
$\forall n\in\mathbb{N},\ v_{n+1}=q\cdot v_{n}$}{\Large \par}

\selectlanguage{french}%
\textbf{\Large{}$\bullet$ Soit $V$ la s�rie associ�e, i.e. la suite
d�finie sur $\mathbb{N}$ par : 
\[
\begin{cases}
V_{0}=v_{0}\\
\forall n\in\mathbb{N},\ V_{n+1}=V_{n}+v_{n+1}
\end{cases}
\]
}{\Large \par}

\textbf{\Large{}\vspace{-10mm}
}{\Large \par}

\textbf{\Large{}Ce qu'on note usuellement sous la forme : ${\displaystyle V_{n}=\sum_{k=0}^{n}v_{k}}$\vspace{2mm}
}{\Large \par}

\textbf{\Large{}$\bullet$ On cherche a exprimer $V_{n}$ }\textbf{\textit{\Large{}en
fonction}}\textbf{\Large{} de $n\in\mathbb{N}$.}{\Large \par}

\textbf{\Large{}$\bullet$ Par distributivit� du produit sur les sommes,
on a, pour tout $n\in\mathbb{N}$, la suite d'�galit�s suivante :
\[
\begin{aligned}(1-q)V_{n} & =V_{n}-q\cdot V_{n}\\
 & =\sum_{k=0}^{n}v_{k}-q\cdot\sum_{k=0}^{n}v_{k}\\
 & =\sum_{k=0}^{n}v_{k}-\sum_{k=0}^{n}q\cdot v_{k}\\
 & =\sum_{k=0}^{n}v_{k}-\sum_{k=0}^{n}v_{k+1}\\
 & =\sum_{k=0}^{n}v_{k}-\sum_{k=1}^{n+1}v_{k}\\
 & =\left(\sum_{k=0}^{n}v_{k}-\sum_{k=0}^{n}v_{k}\right)+\left(v_{0}-v_{n+1}\right)\\
 & =v_{0}-v_{n+1}
\end{aligned}
\]
}{\Large \par}

\textbf{\Large{}\vspace{-4mm}
}{\Large \par}

\textbf{\Large{}$\bullet$ En d�finitive, nous avons ${\displaystyle \forall n\in\mathbb{N},\ V_{n}=\frac{v_{0}-v_{n+1}}{1-q}}$}{\Large \par}

\begin{onehalfspace}
\bordure
\end{onehalfspace}
\selectlanguage{english}%

\end{document}
